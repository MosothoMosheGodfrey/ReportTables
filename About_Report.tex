\documentclass{article}
\usepackage{amsmath}
\usepackage{graphicx}
\usepackage{hyperref}
\usepackage{geometry}

\title{Fuel Industry Insights Report}
\author{Moshe G. Mosotho}
\date{20 February 2025}

% Set the page dimensions
\geometry{
	a4paper,        % Paper size
	left=20mm,      % Left margin
	right=20mm,     % Right margin
	top=15mm,       % Top margin
	bottom=25mm,   % Bottom margin
	headheight=15pt, % Header height
	headsep=10pt    % Space between header and text
}


\begin{document}
	
	\maketitle
	
	\vspace{60pt}
	
	\section*{Target Audience}
	The report is designed for:
	\begin{itemize}
		\item \textbf{Consumers}: Individuals seeking to understand how fuel price trends impact their personal financial planning and budgeting.
		\item \textbf{Fuel Retailers}: Businesses needing to adapt their strategies in response to changes in fuel consumption patterns and pricing dynamics.
		\item \textbf{Banks}: Financial institutions aiming to understand how fuel price changes affect transaction volumes, customer engagement, and overall economic activity.
	\end{itemize}
	
	\newpage
	
	\section*{Overview}
	This report provides strategic insights into the fuel industry, focusing on fluctuating fuel prices and their broader economic implications for consumers and businesses. It is structured into two key sections:
	
	\begin{enumerate}
		\item Part A: Fuel Price Changes
		\item Part B: Fuel Tank Capacity and Consumption
	\end{enumerate}
	
	\section*{Part A: Fuel Price Changes}
	\begin{itemize}
		\item \textbf{Market Dynamics}: This section explores anticipated fluctuations in fuel prices across various fuel types, including petrol, diesel, and illuminating paraffin. It highlights the impact of these changes on both retail and wholesale markets.
		\item \textbf{Economic Implications}: Changes in fuel prices are influenced by macroeconomic factors such as currency exchange rates. A stronger local currency can lead to lower fuel prices, potentially reducing inflation and providing economic relief to consumers. Again, rising fuel prices can strain consumer budgets, affecting spending power and economic activity.
	\end{itemize}
	
	\section*{Part B: Fuel Tank Capacity and Consumption}
	\begin{itemize}
		\item \textbf{Operational Costs}: This part assesses the financial impact of fuel price changes on consumers by estimating the cost of refueling vehicles with different tank capacities. It also examines fuel consumption patterns across various vehicle types, providing insights into how fuel efficiency affects travel costs.
		\item \textbf{Consumer Behavior}: Decreasing fuel prices can lead to increased consumer spending on other things. Again, rising fuel prices may prompt consumers to seek alternative transportation, and thereby impacting our revenues as tansactional volumes, at
		e.g. fuel stations, toll-gates, cross country boarders, are going to decline.
	\end{itemize}
	
	\section*{Impact on ABSA and Customers}
	\begin{itemize}
		\item \textbf{Transaction Volumes}: Fuel price volatility can significantly affect ABSA's transactional volumes, especially at fuel stations where point-of-sale transactions are common, thereby impacting revenue.
		\item \textbf{Customer Engagement}: Fluctuations in fuel prices can alter consumer behavior, leading to shifts in spending patterns and financial priorities. ABSA must adapt its product offerings and customer engagement strategies to meet these changing needs.
		\item \textbf{Economic Activity}: Fuel prices are a critical component of the consumer price index. Changes in fuel prices can impact interest rates and broader economic policies, affecting both ABSA's operations and consumer finances.
	\end{itemize}
	
	\section*{Improvements}
	\begin{enumerate}
		\item \textbf{Interactive Tools}: Develop interactive features like calculators for personalized financial insights and Power BI for dynamic reporting.
		\begin{itemize}
			\item \textbf{AVAF Calculator}: ABSA can integrate fuel price data into its financial calculator on the website to help customers estimate the impact of fuel costs on their budgets and potential cars to buy. This tool can assist in planning for vehicle financing, maintenance, and overall financial management.
			\item \textbf{Power BI Application}: To provide dynamic and interactive reporting capabilities, Power BI can be used to as a replacement for the current reporting format. This will enable near-real-time data analysis and visualization, allowing stakeholders to monitor fuel price trends, consumer behavior, and transaction volumes. %Power BI can help identify patterns and insights that inform strategic decision-making and enhance customer engagement through personalized financial insights.
		\end{itemize}
		
		\item \textbf{Banking Strategies}: Integrate Highlander data and suggest innovative solutions, such as digital banking promotion and fuel-related discounts, to mitigate reduced transaction volumes.
		
		\item \textbf{Oil Price Insights}: Include analysis of crude oil prices to provide valuable insights into the underlying factors driving fuel price changes, helping stakeholders anticipate market shifts and make informed decisions.
	\end{enumerate}
	
\end{document}
